%%%%%%%%%%%%%%%%%%%%%%%%%%%%%%%%%%%%%%%%%
% Programming/Coding Assignment
% LaTeX Template
%
% This template has been downloaded from:
% http://www.latextemplates.com
%
% Original author:
% Ted Pavlic (http://www.tedpavlic.com)
%
% Note:
% The \lipsum[#] commands throughout this template generate dummy text
% to fill the template out. These commands should all be removed when 
% writing assignment content.
%
% This template uses a Perl script as an example snippet of code, most other
% languages are also usable. Configure them in the "CODE INCLUSION 
% CONFIGURATION" section.
%
%%%%%%%%%%%%%%%%%%%%%%%%%%%%%%%%%%%%%%%%%

%----------------------------------------------------------------------------------------
%	PACKAGES AND OTHER DOCUMENT CONFIGURATIONS
%----------------------------------------------------------------------------------------

\documentclass{article}

\usepackage{fancyhdr} % Required for custom headers
\usepackage{lastpage} % Required to determine the last page for the footer
\usepackage{extramarks} % Required for headers and footers
\usepackage[usenames,dvipsnames]{color} % Required for custom colors
\usepackage{graphicx} % Required to insert images
\usepackage{listings} % Required for insertion of code
\usepackage{courier} % Required for the courier font
\usepackage{lipsum} % Used for inserting dummy 'Lorem ipsum' text into the template
\usepackage{amsmath}
\usepackage{amsfonts}
\usepackage{tikz}
\usepackage{url}


% Margins
\topmargin=-0.45in
\evensidemargin=0in
\oddsidemargin=0in
\textwidth=6.5in
\textheight=9.0in
\headsep=0.25in

\linespread{1.1} % Line spacing

% Set up the header and footer
\pagestyle{fancy}
\lhead{\hmwkClass\ : \hmwkTitle} % Top left header
\chead{} % Top center head
\rhead{\firstxmark} % Top right header
\lfoot{\lastxmark} % Bottom left footer
\cfoot{} % Bottom center footer
\rfoot{Page\ \thepage\ of\ \protect\pageref{LastPage}} % Bottom right footer
\renewcommand\headrulewidth{0.4pt} % Size of the header rule
\renewcommand\footrulewidth{0.4pt} % Size of the footer rule

\setlength\parindent{0pt} % Removes all indentation from paragraphs

%----------------------------------------------------------------------------------------
%	CODE INCLUSION CONFIGURATION
%----------------------------------------------------------------------------------------

\definecolor{MyDarkGreen}{rgb}{0.0,0.4,0.0} % This is the color used for comments
\lstloadlanguages{Perl} % Load Perl syntax for listings, for a list of other languages supported see: ftp://ftp.tex.ac.uk/tex-archive/macros/latex/contrib/listings/listings.pdf
\lstset{language=Perl, % Use Perl in this example
        frame=single, % Single frame around code
        basicstyle=\small\ttfamily, % Use small true type font
        keywordstyle=[1]\color{Blue}\bf, % Perl functions bold and blue
        keywordstyle=[2]\color{Purple}, % Perl function arguments purple
        keywordstyle=[3]\color{Blue}\underbar, % Custom functions underlined and blue
        identifierstyle=, % Nothing special about identifiers                                         
        commentstyle=\usefont{T1}{pcr}{m}{sl}\color{MyDarkGreen}\small, % Comments small dark green courier font
        stringstyle=\color{Purple}, % Strings are purple
        showstringspaces=false, % Don't put marks in string spaces
        tabsize=5, % 5 spaces per tab
        %
        % Put standard Perl functions not included in the default language here
        morekeywords={rand},
        %
        % Put Perl function parameters here
        morekeywords=[2]{on, off, interp},
        %
        % Put user defined functions here
        morekeywords=[3]{test},
       	%
        morecomment=[l][\color{Blue}]{...}, % Line continuation (...) like blue comment
        numbers=left, % Line numbers on left
        firstnumber=1, % Line numbers start with line 1
        numberstyle=\tiny\color{Blue}, % Line numbers are blue and small
        stepnumber=5 % Line numbers go in steps of 5
}

% Creates a new command to include a perl script, the first parameter is the filename of the script (without .pl), the second parameter is the caption
\newcommand{\perlscript}[2]{
\begin{itemize}
\item[]\lstinputlisting[caption=#2,label=#1]{#1.pl}
\end{itemize}
}

%----------------------------------------------------------------------------------------
%	DOCUMENT STRUCTURE COMMANDS
%	Skip this unless you know what you're doing
%----------------------------------------------------------------------------------------

% Header and footer for when a page split occurs within a problem environment
\newcommand{\enterProblemHeader}[1]{
\nobreak\extramarks{#1}{#1 continued on next page\ldots}\nobreak
\nobreak\extramarks{#1 (continued)}{#1 continued on next page\ldots}\nobreak
}

% Header and footer for when a page split occurs between problem environments
\newcommand{\exitProblemHeader}[1]{
\nobreak\extramarks{#1 (continued)}{#1 continued on next page\ldots}\nobreak
\nobreak\extramarks{#1}{}\nobreak
}

\setcounter{secnumdepth}{0} % Removes default section numbers
\newcounter{homeworkProblemCounter} % Creates a counter to keep track of the number of problems

\newcommand{\homeworkProblemName}{}
\newenvironment{homeworkProblem}[1][Problem \arabic{homeworkProblemCounter}]{ % Makes a new environment called homeworkProblem which takes 1 argument (custom name) but the default is "Problem #"
\stepcounter{homeworkProblemCounter} % Increase counter for number of problems
\renewcommand{\homeworkProblemName}{#1} % Assign \homeworkProblemName the name of the problem
\section{\homeworkProblemName} % Make a section in the document with the custom problem count
\enterProblemHeader{\homeworkProblemName} % Header and footer within the environment
}{
\exitProblemHeader{\homeworkProblemName} % Header and footer after the environment
}

\newcommand{\problemAnswer}[1]{ % Defines the problem answer command with the content as the only argument
\noindent\framebox[\columnwidth][c]{\begin{minipage}{0.98\columnwidth}#1\end{minipage}} % Makes the box around the problem answer and puts the content inside
}

\newcommand{\homeworkSectionName}{}
\newenvironment{homeworkSection}[1]{ % New environment for sections within homework problems, takes 1 argument - the name of the section
\renewcommand{\homeworkSectionName}{#1} % Assign \homeworkSectionName to the name of the section from the environment argument
\subsection{\homeworkSectionName} % Make a subsection with the custom name of the subsection
\enterProblemHeader{\homeworkProblemName\ [\homeworkSectionName]} % Header and footer within the environment
}{
\enterProblemHeader{\homeworkProblemName} % Header and footer after the environment
}

%----------------------------------------------------------------------------------------
%	NAME AND CLASS SECTION
%----------------------------------------------------------------------------------------

\newcommand{\hmwkTitle}{Assignment\ \#1} % Assignment title
\newcommand{\hmwkDueDate}{Tuesday,\ September\ 20,\ 2021 at 11:59pm PST} % Due date
\newcommand{\hmwkClass}{Topics in Artificial Intelligence (CPSC 532S)} % Course/class
\newcommand{\hmwkClassTime}{11:00am} % Class/lecture time
\newcommand{\hmwkClassInstructor}{Jones} % Teacher/lecturer
\newcommand{\hmwkAuthorName}{John Smith} % Your name

%----------------------------------------------------------------------------------------
%	TITLE PAGE
%----------------------------------------------------------------------------------------

\title{
\textmd{\textbf{\hmwkClass:\ \hmwkTitle}}\\
\normalsize\vspace{0.1in}\small{Due\ on\ \hmwkDueDate}\\
}

\date{} % Insert date here if you want it to appear below your name

%----------------------------------------------------------------------------------------

\begin{document}

\maketitle

\vspace{-0.2in}
In this assignment you will get hands on experience with the basic operations, processing and inner-workings 
of traditional deep learning libraries. There are many deep learning libraries that allow easy creation of complex 
neural network architectures. In future assignments, for example, we will use {\bf PyTorch} to do this (other popular 
ones are Keras, TensorFlow and Theano). However, those libraries in an interest of �ease of use� often abstract 
a lot of detail. The basics learned in this assignment will give you hands on experience on how they work �under 
the hood� and give you skills necessary to implement new types of layers and whole architectures (if necessary) 
and to think through corresponding algorithmic and computational issues. 

%----------------------------------------------------------------------------------------
%	PROBLEM 1
%----------------------------------------------------------------------------------------

% To have just one problem per page, simply put a \clearpage after each problem

\begin{homeworkProblem}[Problem 1 (40 points)]
The key to learning in deep neural networks is ability to compute the derivative of a vector function $\mathbf{f}$ with respect 
to the parameters of the deep neural network. Computing such derivatives as closed-form expressions for complex 
functions $\mathbf{f}$ is difficult and computationally expensive. Therefore, instead, deep learning packages define 
computational graphs and use automatic differentiation algorithms (typically backpropagation) to compute gradients 
progressively through the network using the chain rule. \\


Let us define the following vector functions:
%
\begin{eqnarray}
y_1 & = & \mathbf{f}_1(x_1, x_2) = e^{2x_2} + x_1 sin(3 x_2^2) \\
y_2 & = & \mathbf{f}_2(x_1, x_2) = x_1 x_2 + \sigma(x_2),
\end{eqnarray}
%
where $\sigma(\cdot)$ denotes the standard sigmoid function. This is equivalent to a network with two inputs 
$\mathbf{x} = \begin{bmatrix} x_1 \\ x_2 \end{bmatrix}$ and two outputs 
$\mathbf{y} = \begin{bmatrix} y_1 \\ y_2 \end{bmatrix} = \mathbf{f}(\mathbf{x})$ and a set of intermediate 
layers (note that bold indicates vectors). 
%
\begin{itemize}
\item[(a)] Draw a computational graph. Computational graph should be at the level of elemental mathematical operations (multiplication, square, sine, etc.) and constants/variables. 
\item[(b)] Draw backpropagation graph, based on computational graph above. 
\item[(c)] Compute the value of $\mathbf{f}(\mathbf{x})$ at $\mathbf{x} = \begin{pmatrix} 2 \\  1 \end{pmatrix}$ using forward propagation. Please do not just plug in numbers directly into Eq. (1) and Eq. (2), this is not the intent here.
\item[(d)] At $\mathbf{x} = \begin{pmatrix} 2 \\  1 \end{pmatrix}$, compute Jacobian using forward mode auto-differentiation. Show all steps.
\item[(e)] At $\mathbf{x} = \begin{pmatrix} 2 \\  1 \end{pmatrix}$, compute Jacobian using backward mode auto-differentiation. Show all steps.
\end{itemize}
%
Remember that Jacobian is defined as $\frac{\partial \mathbf{f}}{\partial \mathbf{x}}$. 

\vspace{0.1in}
{\bf Note \#1:} The goal here is {\bf \underline{not}} the final result, it could be obtained easily by any symbolic differentiation package 
(Mathematica, Matlab, etc.) and you will loose points if this is what you do. The goal is to show that you understand 
how backpropagation actually works and can do this by hand. This will be a bit painful, but in fact that is part of the 
point: ability of AutoDiff algorithms to do this automatically is a huge benefit. 

\vspace{0.1in}
{\bf Note \#2:} I am providing the LaTex template for the assignment. You are welcome to do this part directly in LaTex and hand in a PDF. 
However, this is not required. I have had students turn in photos of pages where the math is done by hand. As long as it is legible, that's OK. 

%----------------------------------------------------------------------------------------
%	PROBLEM 2
%----------------------------------------------------------------------------------------

\end{homeworkProblem}

\begin{homeworkProblem}[Problem 2 (15 points)]

Consider a neural network with N input units denoted by $\mathbf{x} \in \mathbb{R}^N$ vector, M output units denoted by $\mathbf{y} \in \mathbb{R}^M$ vector, and K hidden units denoted by $\mathbf{h} \in \mathbb{R}^K$ vector. The hidden layer has an activation function $\sigma$ (e.g., a sigmoid or ReLU).  The resulting equations that govern the behavior of this simple network are:

\begin{eqnarray*}
\mathbf{z} & = & \mathbf{W}^{(1)} \mathbf{x} + \mathbf{b}^{(1)}  \\
\mathbf{h} & = & \sigma(\mathbf{z})  \\
\mathbf{y} & = & \mathbf{W}^{(2)} \mathbf{h} + \mathbf{b}^{(2)}
\end{eqnarray*}

The loss function $\mathcal{L}$ involves an L2 error on prediction $\mathcal{E}$ plus a regularizer $\mathcal{R}$ as follows:
 
\begin{eqnarray*}
\mathcal{L} & = & \mathcal{E} + \mathcal{R}  \\
\mathcal{R} & = &  \mathbf{r}^T \mathbf{h} \\
\mathcal{E} & = & \frac{1}{2} || \mathbf{y} - \mathbf{s} ||^2
\end{eqnarray*}
 
where $\mathbf{r}$ and $\mathbf{s}$ are given (e.g., $\mathbf{s}$ is  a ground truth prediction for $\mathbf{x}$). 

\begin{itemize}
\item[(a)] Draw the computational graph relating $\mathbf{x}$, $\mathbf{z}$, $\mathbf{h}$,  $\mathbf{y}$,  $\mathcal{L}$, $\mathcal{E}$, $\mathcal{R}$. Note, in this case we expect nodes to be representing collections of values, not each neuron individually. 
\item[(b)] Derive the backpropagation equations for computing $\frac{\partial \mathcal{L}}{\partial \mathbf{x}}$. Please use $\sigma'$ to denote the derivative of  the activation function. 
\end{itemize}

{\small {\bf  Credit:} Problem 2 is adopted from University of Toronto's CSC421/2516 Problem Set 1.}
\end{homeworkProblem}


%----------------------------------------------------------------------------------------
%	PROBLEM 3
%----------------------------------------------------------------------------------------

\begin{homeworkProblem}[Problem 3 (80 points)]

The second half of the assignment requires Google Colab and is distributed as a notebook. All instructions are given
in the notebook itself. The assignment does not require any libraries beyond vanilla Python. Once the assignment is done
hand in both the written portion and the programming portion using {\tt Canvas}. When handing in programming assignment
make sure you hand in all the code and the notebook contains the output of executed code. 

\vspace{0.1in}
Those who never used Python Jupyter, can find many tutorials online, including on YouTube, 

~~~~~~~~~~~ e.g., \url{https://www.youtube.com/watch?v=HW29067qVWk}

\noindent
or contact TAs for help. 

\end{homeworkProblem}

%----------------------------------------------------------------------------------------

\end{document}